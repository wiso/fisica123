\begin{savequote}[6cm]
  ``Per chi volesse una prova che i fisici sono umani, la prova sta nell'idiozia di tutte le unità di misura differenti usate per misurare l'energia''
  \qauthor{R.P. Feynman}
\end{savequote}
\chapter{Unità di misura\index{unità di misura}\label{unita_di_misura_appendice}}
\section{Definizioni}
\begin{description}
  \item[quantità in senso generale] Una quantità in generale è una proprietà attribuita ad un fenomeno, corpo o sostanza che può essere quantificata e che può essere attribuita a un fenomeno, corpo o sostanza particolare. Esempio la carica elettrica o la massa.
  \item[quantità in senso particolare] Una quantità in senso particolare è una proprietà quantificabile attribuita a un particolare fenomeno, corpo e sostanza. Esempio la carica del protone o la massa della Luna.
  \item[quantità fisica] Una quantità fisica è una quantità che può essere usata in equazioni matematiche della scienza e della tecnologia.
  \item[unità] Una unità è una particolare quantità fisica, definita e adottata per convenzione, grazie alla quale altre particolari quantità della stessa natura possono essere comparate e espresse con un valore.
  \item[valore di una quantità fisica] Il valore di una quantità fisica è l'espressione quantitativa di una grandezza particolare come il prodotto di un numero e una unità, essendo il numero il valore. Quindi, il valore di una quantità fisica particolare dipende dall'unità con la quale è espressa. In modo più formale il valore di una quantità $A$ può essere scritta come $A={A}[A]$, dove ${A}$ è il valore numerico di $A$ espresso nelle unità $[A]$. Il valore numerico è quindi: ${A}=A/[A]$. Per eliminare possibili incomprensioni negli assi dei grafici è più corretto scrivere $t/\si{\celsius})$ al posto di $t(\si{\celsius})$.
\end{description}
\section{SI}
\begin{table}[ht]
  \centering
  \begin{tabular}{llc}
    \toprule
    grandezza                 & nome       & simbolo        \\
    \midrule
    lunghezza                 & metro      & \si{\meter}    \\
    massa                     & kilogrammo & \si{\kilogram} \\
    tempo                     & secondo    & \si{\second}   \\
    corrente elettrica        & ampere     & \si{\ampere}   \\
    temperatura termodinamica & kelvin     & \si{\kelvin}   \\
    quantità di materia       & mole       & \si{\mole}     \\
    intensità luminosa        & candela    & \si{\candela}  \\
    \bottomrule
  \end{tabular}
  \caption{Unità SI fondamentali}
\end{table}
\subsection{secondo}
In principio il secondo fu riferito al giorno solare, come $1/86400$ del giorno solare medio. Purtroppo le osservazioni mostrarono che la rotazione della Terra era lievemente irregolare. Nel 1960 l'undicesimo CGPM adottò una definizione dall' Astronomica Unione che si basava sull'anno tropicale. Nel 1967 il tredicesimo CGPM decise di rimpiazzare la definizione di secondo con quella attuale:
\begin{definizioneunita}
  il secondo è la durata di ${9\,192\,631\,770}$ periodi della radiazione corrispondente alla transizione tra due livelli iperfini allo stato terreno del cesio--133.
\end{definizioneunita}
\subsection{metro}
L'origine del metro risale al diciottesimo secolo. A quel tempo c'erano due approcci in competizione. Uno definiva il metro come la lunghezza del pendolo avente come semiperiodo un secondo; l'altro suggeriva di definire il metro come la decimilionesima lunghezza del meridiano terrestre lungo un quadrante (un quarto della circonferenza terrestre). Nel 1971, poco dopo la rivoluzione francese, l'Accademia Francese delle Scienze decise per la definizione in base al meridiano, in quanto la forza di gravità variava da punto a punto sulla superficie della Terra, modificando il periodo del pendolo e un metro fu definito come un un milionesimo della lunghezza del meridiano passante per Parigi e l'equatore. Tuttavia il primo prototipo era più corto di $\SI{0.2}{\milli\meter}$ perché i ricercatori ignorarono lo schiacciamento ai poli. Tuttavia il campione di platino--iridio fu preso come campione. Nel 1889, un nuovo prototipo internazionale fu costruito con maggior precisione e tenuto alla temperatura di fusione del ghiaccio. Nel 1927, il metro fu definito in maniera più precisa come la distanza a $\SI{0}{\celsius}$ tra gli assi di due linee centrali incise su una barra di platino--iridio. Nel 1960 fu definito il metro in base alla lunghezza d'onda della radiazione del cripto--86. Nel 1983 il CGPM cambiò l'ultima definizione con quella attuale:
\begin{definizioneunita}
  il metro è la lunghezza della distanza compiuta dalla luce nel vuoto in $\SI[parse-numbers = false]{1/299792458}{\second}$.
\end{definizioneunita}
\subsection{Chilogrammo}
Alla fine del diciottesimo secolo il chilogrammo era definito come la massa di $\SI{1}{\deci\meter\cubed}$ di acqua. Nel 1889 fu adottato l'uso di un prototipo di platino--iridio.
\begin{definizioneunita}
  il chilogrammo è l'unita di massa; è uguale alla massa del prototipo internazionale di chilogrammo
\end{definizioneunita}
\subsection{ampere}
L'ampere fu introdotto nel 2893 dal Congresso Internazionale di Chicago nel 1893 e poi adottato nella Conferenza Internazionale di Londra nel 1908.
\begin{definizioneunita}
  l'ampere è la corrente costante che, se mantenuta in due conduttori rettilinei paralleli di infinita lunghezza, di sezione circolare trascurabile, e posti alla distanza di $\SI{1}{\meter}$ l'uno dall'altro nel vuoto, produce tra di essi una forza pari a $\SI{2E-7}{\newton}$ per metro di lunghezza.
\end{definizioneunita}

In questo modo $\mu_0=4\SI[parse-numbers = false]{\pi e-7}{\henry\per\meter}$ esattamente.


\begin{table}[ht]
  \centering
  \begin{tabular}{llll}
    \toprule
    nome        & unità SI                                                           & simbolo         & unità equivalenti                       \\
    \midrule
    radiante    & \si{\meter\per\meter}\,=1                                          & \si{\radian}    & \si{\meter\per\meter}                   \\
    steradiante & \si{\meter\squared\per\meter\squared}\,=1                          & \si{\steradian} & \si{\meter\per\meter\squared}           \\
    hertz       & \si{\per\second}                                                   & \si{\hertz}     & \si{\per\second}                        \\
    newton      & \si{\meter\kilogram\per\second\squared}                            & \si{\newton}    & \si{\meter\kilogram\per\second\squared} \\
    pascal      & \si{\per\meter\kilogram\per\second\squared}                        & \si{\pascal}    & \si{\newton\per\meter\squared}          \\
    joule       & \si{\meter\squared\kilogram\per\second\squared}                    & \si{\joule}     & \si{\newton\meter}                      \\
    watt        & \si{\meter\squared\kilogram\per\second\cubed}                      & \si{\watt}      & \si{\joule\per\second}                  \\
    coulomb     & \si{\ampere\second}                                                & \si{\coulomb}   & \si{\ampere\second}                     \\
    volt        & \si{\meter\squared\kilogram\per\second\cubed\per\ampere}           & \si{\volt}      & \si{\watt\per\ampere}                   \\
    farad       & \si{\ampere\squared s^4\per\meter\squared\per\kilogram}            & \si{\farad}     & \si{\coulomb\per\volt}                  \\
    ohm         & \si{\meter\squared\kilogram\per\second\cubed\per\ampere\squared}   & \si{\ohm}       & \si{\volt\per\ampere}                   \\
    siemens     & \si{\ampere^2\second^3\per\kilogram\per\meter^2}                   & \si{\siemens}   & \si{\per\ohm}=\si{\ampere\per\volt}     \\
    weber       & \si{\meter\squared\kilogram\per\second\squared\per\ampere}         & \si{\weber}     & \si{\volt\second}                       \\
    tesla       & \si{\kilogram\per\second\squared\per\ampere}                       & \si{\tesla}     & \si{\weber\per\meter\squared}           \\
    henry       & \si{\meter\squared\kilogram\per\second\squared\per\ampere\squared} & \si{\henry}     & \si{\weber\per\ampere}                  \\
    celsius     & \si{\kelvin}                                                       & \si{\celsius}   & \si{\kelvin}                            \\
    lumen       & \si{\candela\meter\squared\per\meter\squared}                      & \si{\lumen}     & \si{\candela\steradian}                 \\
    lux         & \si{\candela\meter\squared\meter^{-4}}                             & \si{\lux}       & \si{\lumen\per\meter\squared}           \\
    \bottomrule
  \end{tabular}
  \caption{Unità SI derivate con nomi e simboli speciali}
\end{table}

\begin{table}[ht]
  \centering
  \begin{tabular}{lll}
    \toprule
    nome                   & simbolo                 & valore in unità SI                                                                       \\
    \midrule
    minuto (tempo)         & \si{\minute}            & $\SI{1}{\minute} = \SI{60}{\second}$                                                     \\
    ora                    & \si{\hour}              & $\SI{1}{\hour}=\SI{60}{\minute}=\SI{3600}{\second}$                                      \\
    giorno                 & \si{\day}               & $\SI{1}{\day}=\SI{24}{\hour}=\SI{86400}{\second}$                                        \\
    grado                  & \si{\degree}            & $\SI{1}{\degree}=\pi/180\si{\radian}$                                                    \\
    minuto (angolo piano)  & \si{\arcminute}         & $\SI{1}{\arcminute}=\SI[parse-numbers = false]{1/60}{\degree}=\pi/10800\si{\radian}$     \\
    secondo (angolo piano) & \si{\arcsecond}         & $\SI{1}{\arcsecond}=\SI[parse-numbers = false]{1/60}{\arcminute}=\pi/648000\si{\radian}$ \\
    litro                  & \si{\litre},\si{\liter} & $\SI{1}{\litre}=\SI{1}{\liter}=\SI{1}{\deci\meter\cubed}=\SI{1E-3}{\meter\cubed}$        \\
    tonnellata             & \si{\tonne}             & $\SI{1}{\tonne}=\SI{1E3}{\kilogram}$                                                     \\
    neper                  & \si{\neper}             & $\SI{1}{\neper}=1$                                                                       \\
    bel                    & \si{\bel}               & $\SI{1}{\bel}=(1/2)\log_{10}(\si{\neper})$                                               \\
    \bottomrule
  \end{tabular}
  \caption{Unità accettate per l'uso con l'SI}
\end{table}

\ctable[caption = Unità accettate nell'SI i cui valori in SI sono determinati sperimentalmente,
  pos = ht,
  label = tab:SI05,
  width = 10cm,
  botcap,]
{llc}
{
  \tnote[a]{l'elettronvolt è l'energia cinetica acquistata da un elettrone passando da una differenza di potenziale di \SI{1}{\volt} nel vuoto; $\SI{1}{\electronvolt}=1.602\,177\,33\times 10^{-19}\pm 0.000\,000\,49 \si{\joule}$}
  \tnote[b]{l'unità di massa atomica è equivalente a $1/12$ della massa di un atomo di $^{12}\mathrm{C}$; $\SI{1}{\atomicmassunit}=1.660\,540\,2\pm 0.000\,001\,0 \times 10^{-27}\si{\kilogram}$}
  \tnote[c]{l'unità astronomica è tale che quando è usata per descrivere il moto dei pianeti nel sistema solare la costanze gravitazionale eliocentrica vale $(0.017\,202\,098\,95)^2\si{\astronomicalunit\cubed\per\day\squared}$}
}
{\toprule
  nome&simbolo&definizione\\
  \midrule
  elettronvolt&\si{\electronvolt}&\tmark[a]\\
  unità di massa atomica&\si{\atomicmassunit}&\tmark[b]\\
  unità astronomica&\si{\astronomicalunit}&\tmark[c]\\
  \bottomrule
}





\begin{table}[ht]
  \centering
  \begin{tabular}{lll}
    \toprule
    nome       & simbolo        & valore in unità SI                                                       \\
    \midrule
    angstr\"om & \si{\angstrom} & $\SI{1}{\angstrom}=\SI{0.1}{\nano\meter}=\SI{1E-10}{\meter}$             \\
    ettaro     & \si{\hectare}  & $\SI{1}{\hectare}=\SI{1}{\hecto\meter\squared}=\SI{1E4}{\meter\squared}$ \\
    bar        & \si{\bar}      & $\SI{1}{\bar}=\SI{1E5}{\pascal}$                                         \\
    \bottomrule
  \end{tabular}
  \caption{unità in uso temporaneo nell'SI}
\end{table}

\begin{table}[ht]
  \centering
  \begin{tabular}{llcl}
    \toprule
    Grandezza         & Nome                      & Unità                          & Unità equivalenti \\
    \midrule
    superficie        & metro quadrato            & \si{\meter\squared}                                \\
    volume            & metro cubo                & \si{\cubic\meter}                                  \\
    frequenza         & hertz                     & \si{\hertz}                    & \si{\per\second}  \\
    densità           & kilogammo al metro cubo   & \si{\kilogram\per\cubic\meter}                     \\
    velocità          & metro al secondo          & \si{\metre\per\second}                             \\
    velocità angolare & radiante al secondo       & \si{\radian\per\second}        & \si{\per\second}  \\
    accelerazione     & metro al secondo quadrato & \si{\meter\per\second\squared} &                   \\
    \bottomrule
  \end{tabular}
\end{table}

\section{Elettrodinamica}
\begin{table}[ht]
  \centering
  \begin{tabular}{llcl}
    \toprule
    Grandezza                      & Nome    & Unità                           & Unità equivalenti                                        \\
    \midrule
    Carica                         & coulomb & \si{\coulomb}                   & \si{\ampere\second}                                      \\
    Densità di carica superficiale &         & \si{\coulomb\per\meter\squared} & \si{\ampere\second\per\meter\squared}                    \\
    Densità di carica volumetrica  &         & \si{\coulomb\per\meter\cubed}   & \si{\ampere\second\per\meter\cubed}                      \\
    Potenziale elettrostatico      & volt    & \si{\volt}                      & \si{\meter\squared\kilogram\per\second\cubed\per\ampere} \\
    \bottomrule
  \end{tabular}
\end{table}
