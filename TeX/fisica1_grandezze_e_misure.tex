\chapter{Grandezze e misure}
\minitoc
\section{\index{grandezze}Grandezze}
Le grandezze sono enti di cui si occupa la fisica. Esse devono essere misurabili attraverso un metodo operativo, inoltre richiedono delle unità di misura con le quali avviene il confronto.
\newline\par
Le grandezze si dividono in:
\begin{itemize}
  \item\index{grandezze!scalari}grandezze scalari (numero,unità)
  \item\index{grandezze!vettoriali}grandezze vettoriali
  (numero,unità,direzione,verso)=(vettore,unità)
  \item\index{grandezze!tensoriali}grandezze tensoriali
\end{itemize}
Le grandezze hanno una dimensione data dall'unità di misura. Esistono grandezze adimensionali (numeri puri), o più raramente dei vettori, tensori puri. L'unità di misura deve essere definita rispetto a qualcosa di invariante e riproducibile. La lunghezza è da considerarsi scalare, mentre la posizione vettoriale.
%\section[Unità di misura]{\index{unità di misura}Unità di misura}
%\subsection{\index{unità di misura!fondamentali}Unità fondamentali}
%\begin{minipage}[c]{\textwidth}
%Nel sistema internazionale (SI) sono definite 7 unità di misura fondamentali:
%\vspace{0.2 cm}
%\begin{small}
%\begin{tabular}{p{3.1cm}lcp{5.4cm}}
%\hline Grandezza & Nome & Simbolo & Definizione\\ \hline
%\index{lunghezza}lunghezza & \index{metro}metro &\metre& ``\ldots la lunghezza è la distanza percorsa dalla %luce nel vuoto in 1/299792458 di secondo''\\\begin{s}

%\index{massa}massa & \index{kilogrammo}kilogrammo &\kilogram&``\ldots questo prototipo (un particolare %cilindro di pla\-ti\-no--i\-ri\-dio) potrà d'ora in poi essere considerato l'unità di massa''\\
%\index{tempo}tempo & \index{secondo}secondo &\second &``\ldots la durata di $9192631.770$ periodi della %radiazione corrispondente alla transizione tra i due livelli iperfini dello stato fondamentale dell'atomo di %cesio 133''\\
%\index{corrente!elettrica}corrente elettrica & \index{ampere}ampere &\ampere&``\ldots quella corrente %costante che, passando in due conduttori paralleli rettilinei infinitamente lunghi, di sezione circolare %trascurabile, posti a \SI{1}{\meter} di distanza nel vuoto produce tra i due conduttori una forza di %$\SI{2E-7}{\newton}$ per metro di lunghezza''\\
%\index{temperatura termodinamica}temperatura termodinamica & \index{kelvin}kelvin & \kelvin &``\ldots la %frazione $1/273.16$ della temperatura termodinamica del punto triplo dell'acqua''\\
%\index{quantità di sostanza}quantità di sostanza&\index{mole}mole&\mole&``\ldots la quantità di
%sostanza di un sistema che contiene tale entità elementari quanti sono gli atomi contenuti in %$\SI{0.012}{\kilogram}$ di carbonio 12''\\
%%\index{intensità!luminosa}intensità luminosa& \index{candela}candela&\candela&``\ldots l'intensità luminosa, %in una data direzione, di una sorgente che emette una radiazione monocromatica di frequenza %$\SI{540E12}{\hertz}$ e la cui intensità energetica in tale direzione è di %$1/683\SI{1}{\watt\per\steradian}$''
%\\
%\hline
%\end{tabular}
%\end{small}
%\end{minipage}

\subsection{\index{unità di misura!prefissi}Prefissi}
Spesso si usano i seguenti prefissi:
\begin{center}
  \begin{tabular}{lcc}
    \hline
    fattore     & prefisso & simbolo          \\
    \hline
    \num{1E24}  & yotta    & \si{\yotta\noop} \\
    \num{1E21}  & zetta    & \si{\zetta\noop} \\
    \num{1E18}  & exa      & \si{\exa\noop}   \\
    \num{1E15}  & peta     & \si{\peta\noop}  \\
    \num{1E12}  & tera     & \si{\tera\noop}  \\
    \num{1E9}   & giga     & \si{\giga\noop}  \\
    \num{1E6}   & mega     & \si{\mega\noop}  \\
    \num{1E3}   & kilo     & \si{\kilo\noop}  \\
    \num{1E2}   & etto     & \si{\hecto\noop} \\
    \num{1E1}   & deca     & \si{\deka\noop}  \\
    \num{1E-1}  & deci     & \si{\deci\noop}  \\
    \num{1E-2}  & centi    & \si{\centi\noop} \\
    \num{1E-3}  & milli    & \si{\milli\noop} \\
    \num{1E-6}  & micro    & \si{\micro\noop} \\
    \num{1E-9}  & nano     & \si{\nano\noop}  \\
    \num{1E-12} & pico     & \si{\pico\noop}  \\
    \num{1E-15} & femto    & \si{\femto\noop} \\
    \num{1E-18} & atto     & \si{\atto\noop}  \\
    \num{1E-21} & zepto    & \si{\zepto\noop} \\
    \num{1E-24} & yocto    & \si{\yocto\noop} \\
    \hline
  \end{tabular}
\end{center}
Il femtometro (\si{\femto\meter}) per un puro caso ha lo stesso simbolo e valore del fermi\index{Fermi}\index{fm@\si{\femto\meter}|see{Fermi}}, unità usata in fisica nucleare.
