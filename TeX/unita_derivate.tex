\chapter{Unit� di misura\index{unit� di misura}\label{unita_di_misura_appendice}}
\begin{table}[ht]
\centering
\begin{tabular}{llll}
\hline
nome&unit� SI&simbolo&unit� equivalenti\\
\hline
radiante&\meter\usk\reciprocal\meter\,=1&\rad&\meter\usk\reciprocal\meter\\
steradiante&\meter\squared\usk\meter\rpsquared\,=1&\steradian&\meter\squared\usk\meter\\
hertz&\reciprocal\second&\hertz&\reciprocal\second\\
newton&\meter\usk\kilogram\usk\rpsquare\second&\newton&\meter\usk\kilogram\usk\rpsquare\second\\
pascal&\reciprocal\meter\usk\kilogram\usk\rpsquare\second&\pascal&\newton\usk\rpsquare\meter\\
joule&\meter\squared\usk\kilogram\usk\rpsquare\second&\joule&\newton\usk\meter\\
watt&\meter\squared\usk\kilogram\usk\second\rpcubed&\watt&\joule\usk\reciprocal\second\\
coulomb&\ampere\usk\second&\coulomb&\ampere\usk\second\\
volt&\meter\squared\usk\kilogram\usk\second\rpcubed\usk\reciprocal\ampere&\volt&\watt\usk\reciprocal\ampere\\
\hline
\end{tabular}
\caption{Unit� SI derivate con nomi e simboli speciali}
\end{table}

\begin{table}[ht]
\centering
\begin{tabular}{lll}
\hline
nome&simbolo&valore in unit� SI\\
\hline
minuto (tempo)&\minute&$\unita{1}\minute = \unita{60}\second$\\
ora&\hour&$\unita{1}\hour=\unita{60}\minute=\unita{3600}\second$\\
giorno&\dday&$\unita{1}\dday=\unita{24}\hour=\unita{86400}\second$\\
grado&\degree&$\unita{1}\degree=\unita{(\pi/180)}\rad$\\
minuto (angolo piano)&\arcminute&$\unita{1}\arcminute=\unita{(1/60)}\degree=\unita{(\pi/10800)}\rad$\\
secondo (angolo piano)&\arcsecond&$\unita{1}\arcsecond=\unita{(1/60)}\arcminute=\unita{(\pi/648000)}\rad$\\
litro&\litre,\liter&$\unita{1}\litre=\unita{1}\liter=\unita{1}\deci\meter\cubed=\unita{10^{-3}}\meter\cubed$\\
tonnellata&\tonne&$\unita{1}\tonne=\unita{10^3}\kilogram$\\
neper&\neper&$\unita{1}\neper=1$\\
bel&\bel&$\unita{1}\bel=(1/2)\log_{10}(\neper)$\\
\hline
\end{tabular}
\caption{Unit� accettate per l'uso con l'SI}
\end{table}

\begin{table}[ht]
\centering
\begin{tabular}{lll}
\hline
nome&simbolo&definizione\\
\hline
elettronvolt&\electronvolt&\footnote{l'elettroncvolt � l'energia cinetica acquistata da un elettrone passando da una differenza di potenziale di \unita{1}\volt nel vuoto; $\unita{1}\electronvolt=1.602\,177\,33\times 10^{-19}\pm 0.000\,000\,49\joule$}\\
unit� di massa atomica&\atomicmass&\footnote{l'unit� di massa atomica � equivalente a $1/12$ della massa di un atomo di $^{12}\mathrm{C}$; $\unita{1}\atomicmass=1.660\,540\,2\pm 0.000\,001\,0 \times 10^{-27}\kilogram$}\\
unit� astronomica&\ua&\footnote{l'unit� astronomica � tale che quando � usata per descrivere il moto dei pianeti nel sistema solare la costanze gravitazionale eliocentrica vale $(0.017\,202\,098\,95)^2\ua\cubed\rpsquare\dday$}$\unita{1}\ua)\unita{1.495\,98\times 10^{11}}\meter$\\
\hline
\end{tabular}
\caption{Unit� accettate nell'SI, i cui valori in SI sono determinati sperimentalmente}
\end{table}


\begin{table}[ht]
\centering
\begin{tabular}{lll}
\hline
nome&simbolo&valore in unit� SI\\
\hline
angstr\"om&\angstrom&$\unita{1}\angstrom=\unita{0.1}\nano\meter=\unita{10^{-10}}\meter$\\
ettaro&\hectare&$\unita{1}\hectare=\unita{1}\hecto\meter\squared=\unita{10^4}\meter\squared$\\
\hline
\end{tabular}
\caption{unit� in uso temporaneo nell'SI}
\end{table}

\begin{table}[ht]
\centering
\begin{tabular}{llcl}
\hline
Grandezza&Nome&Unit�&Unit� equivalenti\\
\hline superficie&metro quadrato&\meter\squared\\
volume&metro cubo&\cubic\meter\\
frequenza&hertz&\hertz&\reciprocal\second\\
densit�&kilogammo al metro cubo&\kilogram\per\cubic\meter\\
velocit�&metro al secondo&\metrepersecond\\
velocit� angolare&radiante al secondo&\rad\per\second&\reciprocal\second\\
accelerazione&metro al secondo quadrato&\meter\per\second\squared&\\
\hline
\end{tabular}
\end{table}

\section{Elettrodinamica}
\begin{table}[ht]
\centering
\begin{tabular}{llcl}
\hline
Grandezza&Nome&Unit�&Unit� equivalenti\\
\hline
Carica&coulomb&\coulomb&\ampere\usk\second\\
Densit� di carica superficiale&&\coulomb\per\meter\squared&\ampere\usk\second\usk\meter\rpsquared\\
Densit� di carica volumetrica&&\coulomb\per\meter\cubed&\ampere\usk\second\usk\meter\rpcubed\\
Potenziale elettrostatico&volt&\volt&\meter\squared\usk\kilogram\usk\second\rpcubed\usk\reciprocal\ampere\\
\hline
\end{tabular}
\end{table}