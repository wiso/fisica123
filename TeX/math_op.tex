\DeclareMathOperator{\grad}{grad}
\DeclareMathOperator{\diver}{div}
\DeclareMathOperator{\rot}{rot}
\DeclareMathOperator{\arcsinh}{arcsinh}
\DeclareMathOperator{\arccosh}{arccosh}
\DeclareMathOperator{\arctanh}{arctanh}
\DeclareMathOperator{\erf}{Erf}
\DeclareMathOperator{\sgn}{sgn}


\providecommand{\norm}[1]{\left\lVert#1\right\rVert}
\providecommand{\abs}[1]{\left\lvert#1\right\rvert}
\providecommand{\vecb}[1]{\boldsymbol{#1}}
\providecommand{\ten}[1]{\overset{\Rightarrow}{#1}}
\providecommand{\tens}[1]{\underline{\underline{#1}}}
\providecommand{\media}[1]{\left<#1\right>}
\providecommand{\ver}[1]{\vecb{\hat {#1}}}


\newcommand{\ud}{\mathrm{d}}
\newcommand{\const}{\text{const}}
\newcommand{\riq}{\boxed}
\newcommand{\ve}{\vecb}
\newcommand{\clearemptydoublepage}{\newpage{\pagestyle{empty}\cleardoublepage}}
\newcommand{\giorgi}{4\pi\varepsilon_0}
\newcommand{\e}{\mathrm{e}}
\newcommand{\field}[1]{\mathbb{#1}}

\newcommand*\mygraybox[1]{%
\colorbox{lightgray}{\hspace{1em}#1\hspace{1em}}}

\newenvironment{eqimp}[2][]{%
\setkeys{EmphEqEnv}{#2}%
\setkeys{EmphEqOpt}{box=\mygraybox,#1}%
\EmphEqMainEnv}%
{\endEmphEqMainEnv}